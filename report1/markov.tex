\documentclass[a4paper,12pt,russian]{article}
%\usepackage[T1]{fontenc}
%\usepackage[utf8x]{inputenc}
\usepackage{babel,amsmath}
%\usepackage{lmodern}

\title{Начальные сведения о конструктивной математике А. А. Маркова}
\author{Михайлов Анатолий Андреевич (601-41)}

\begin{document}

\maketitle
%\tableofcontents

\begin{abstract}
Доклад подготовлен студентом первого курса кафедры прикладной математики в сентябре 2024-го года. Цель доклада --- дать неподготовленному читателю начальные сведения о конструктивной математике Андрея Андреевича Маркова младшего в доступной форме, чтобы породить интерес к работам А.А. и фундаментальным математическим теориям в целом. Для этого в докладе затрагиваются проблемы классической математики, нормальные алгорифмы, конструктивная логика, сама конструктивная математика и теория типов. Также, доклад содержит краткую справку о самом А. А. Маркове.
\end{abstract}

\section{Об А. А. Маркове}
\paragraph{Андрей Андреевич Марков младший} (22.09.1903, Санкт-Петербург -- 11.10.1979, Москва) --- сын великого математика Андрея Андреевича Маркова старшего, ученика Пафнутия Львовича Чебышёва \cite{bre}. Получил уникальное домашнее воспитание и образование под руководством отца. Его учили языкам (основными европейскими он отлично владел ещё с юных лет), музыке, рисованию. Как и отец, отлично играл в шахматы. Любил литературу, особенно поэзию. \cite[Том 1, От составителя, пункт 3]{markov} А. А. Марков по своему образованию не был ни математиком, ни логиком (он в 1924 г. окончил физическое отделение физико-математического факультета Ленинградского университета), и тем не менее сумел добиться значительных успехов в теоретической физике, в прикладной геофизике, в небесной механике и даже в химии \cite{nagorny_markov}. В 1933–1955 гг. работал в ЛГУ (профессор с 1936), с 1959 г. профессор МГУ, в 1939–1972 --- в Математическом институте имени В. А. Стеклова АН СССР, с 1972 г. --- в Вычислительном центре АН СССР \cite{bre}. А. A. Марков и вслед за ним его ученики, в том числе и Н. М. Нагорный, следуя ленинградской школе математиков, вместо слова <<алгоритм>> писали <<алгорифм>> \cite{nagorny_markov}. Н. М. Нагорный в \cite[Том 1, От составителя, пункт 1.1]{markov} пишет, что натура А. А. Маркова <<была во многом художественной и даже артистичной>>. В этой же книге в сносках описывается множество забавных моментов, связанных с А. А. Марковым. Н. М. Нагорный также пишет, что А. А. Марков обладал способностью точно планировать свои как краткосрочные, так и долгосрочные действия. Это выражалось как в том, что он работал строго в направлении к желаемому результату, <<как будто его вела невидимая рука>>, так и в том, что он выражал свои мысли очень выверенно, легко для восприятия и точно.

\section{Основа конструктивного направления}
\subsection{Конструктивные процессы и объекты}
\paragraph{}Материал излагаемый в этой главе в основном основан на работе \cite[О логике конструктивной математики]{markov}. Конструктивная математика --- наука о конструктивных процессах, способности их осуществлять и их результатах \cite{markov}. Конструктивный процесс --- действие, в результате которого строится (создаётся) какой-либо объект \emph{(пояснение от автора доклада, не встречается в литературе)}. Понятия конструктивного процесса и конструктивного объекта являются первоначальными \cite{bre}. Согласно \cite{markov}, в определениях конструктивных процессов и объектов <<нет надобности, т.к. каждая математическая теория имеет дело не с конструктивными объектами вообще, а с конструктивными объектами некоторого определённого вида, например со словами в некотором алфавите>>. Т.е., \emph{(по мнению автора доклада)} предлагается характеризовать процессы и объекты (теперь и далее слово <<конструктивный>> будет опускаться) для каждого конкретного случая отдельно.
Примеры конструктивных процессов:
\begin{itemize}
  \item Сборка часов на конвеере. Результат этого процесса, объект --- часы.
  \item Написание 3-ёх чёрточек на пустом листе бумаги. Результат, объект --- материальное тело, состоящее из чернил и бумаги.
  \item Приписывание чёрточки справа к уже имеющимся. Результат как в предыдущем пункте.
\end{itemize}
\paragraph{}При этом, процесс может быть невыполним по ресурсам (например, написать $10^{100^{100}}$ чёрточек: для этого не хватит ни места, ни чернил, ни времени), но такой процесс и его результат всё равно можно рассматривать так же, как и выполнимые процессы. Эта условность называется \emph{абстракцией потенциальной осуществимости}. Она используется и в классической математике, когда, например, рассматривается сумма гигантского (но конечного) количества чисел.
\paragraph{}Однако, есть условность, которую принимает классическое направление, но отвергвет конструктивное: рассмотрение бесконечных объектов как завершённых, неподвижных во времени --- \emph{абстракция актуальной бесконечности}. Например, слово, состоящее из бесконечного количества букв, соответственно имеющее бесконечную длину. В этом примере слово является завершённым статическим (неизменном во времени) объектом. Из-за этой условности в классической математике возникают противоречия (парадоксы), связанные с бесконечными множествами, как, например, парадокс Б. Рассела. Соответственно, отвергая завершённые бесконечные объекты, конструктивное направление избавляется от такого рода парадоксов.

\subsection{Конструктивная логика}
\paragraph{Закон исключённого третьего} это закон классической логики --- истинно или само утверждение, или его отрицание: $P \lor \neg P$. Этот закон позволяет \emph{метод доказательства от противного}: если мы опровергли $\neg P$, значит мы доказали $P$. Из закона исключённого третьего следует закон двойного отрицания: $\neg \neg P \implies P$. Пример на деревянном столе:
\begin{align*}
  P &= \text{стол деревянный}\\
  \neg P &= \text{стол не деревянный}\\
  \neg \neg P &= \text{стол не не деревянный}\\
  \neg \neg P &\implies P\\
  \text{стол не не деревянный} &\implies \text{стол деревянный}
\end{align*}
\paragraph{Конструктивная математическая логика} отличается от классической математической (аристотелевской) логики правилами доказательств существования объектов с нужными свойствами \cite[стр. 9]{markov_con}. В конструктивной математике для доказательства существования объекта достаточно и необходимо предоставить способ его построения (конструкции). Как следствие, нет доказательства от противного. Недостаточно (как это можно делать в классической математике) опровергнуть предположение о том, что не существует искомый объект. Из этого следует отсутствие закона исключённого третьего, потому что если было опровергнуто $\neg P$, это ещё не означает, что доказано $P$. В следствии этого фундаментального отличия, отличаются и другие логические связки. \cite[О логике конструктивной математики, параграф 6]{markov} \cite[стр. 9]{markov_con}

\section{Нормальные алгорифмы}
\paragraph{}При описании алгорифмов на естественных языках возникают некоторые проблемы, поэтому математики разработали свои строгие исскуственные языки, на которых можно записывать алгорифмы. Из широко известных --- это $\lambda$-исчисление Алонзо Чёрча, комбинаторная логика и машина Тьюринга. А. А. Марков тоже разработал подобную систему (формализм) --- \emph{нормальные алгорифмы}. Все эти формализмы функционально идентичны \cite[стр. 10]{markov_con}. Создание таких формализмов позволило:
\begin{itemize}
  \item Передавать алгорифмы с абсолютной точностью (без трактовок).
  \item Упростить и вывести на качественно новый уровень анализ алгоритмов (см. колмогоровская сложность и исчисление большого О).
  \item Автоматизировать (механизировать) вычисление алгорифмов.
\end{itemize}
\paragraph{Вкратце про нормальные алгорифмы} Алфавит --- упорядоченный список символов. Пример: A = (0, |, x, ш). Слово алфавита $x$ --- упорядоченный список, состоящий только из букв этого алфавита. Пример, одно из слов алфавита А: <<|0шшшх00>>. Нормальный алгорифм в алфавите $x$ --- упорядоченный список правил подстановок вида
\begin{align*}
  A &\rightarrow B\\
  &\text{или}\\
  A & \rightarrow\!\cdot
\end{align*}
где $A$ и $B$ --- слова. Правила применяются по порядку: сначала пытаются применить 1-ое правило, и только если оно не подходит, переходят к следующему. Если правило подходит, то первое вхождение $A$ заменяется на $B$ и цикл повторяется с 1-ого правила. Цикл повторяется до тех пор, пока не встретится правило второго типа (завершающее) или не подойдёт ни одно правило. Тогда алгорифм завершается.

\section{Практическое применение}
\paragraph{Доказательства на основе теории типов} это яркий пример практического применения конструктивной математики. Чтобы доказать утверждение, мы буквально конструируем его совмещая объекты нужных типов. Пример написан на языке LF, взят из документации к программе Twelf \cite{twelf}:
\begin{quote}
  nat : type.\\
  z : nat.\\
  s : nat -> nat.\\
  even : nat -> type.\\
  even-z : even z.\\
  even-s : {N:nat} even N -> even (s (s N)).\\
  two-is-even = even-s z even-z.
\end{quote}
Здесь доказательством того, что 2 --- чётное число является объект <<\emph{two-is-even}>>. Он получается путём применения аксиом, которые являются типизироваными функциями, что можно рассматривать как конструктивный процесс, как алгорифм, как программу. Такое соответствие между доказательствами и алгорифмами называется <<соответствие Карри–Ховарда>>.

\begin{thebibliography}{9}
\bibitem{markov}
  А. А. Марков,
  Избранные труды,
  МЦНМО, 2002
\bibitem{markov_alg}
  А. А. Марков, Н. М. Нагорный,
  Теория алгорифмов
  «Наука», 1984, 
\bibitem{markov_con}
  А. А. Марков,
  О конструктивной математике,
  Тр. МИАН СССР, 1962, том 67, 8–14
\bibitem{nagorny_markov}
  Н.М.Нагорный,
  Реализуемостная семантика раннего периода марковского конструктивизма
\bibitem{bre}
  Большая российская энциклопедия,
  https://bigenc.ru/
\bibitem{mem_nagorny}
  М. К. Керимов,
  Памяти Николая Макаровича Нагорного (1928–2007),
  Ж. вычисл. матем. и матем. физ., 2008, том 48, номер 6, 1140–1144
%\bititem{matenc}
%  Математическая энциклопедия,
%  Издательство "Советская Энциклопедия", 1982
\bibitem{twelf}
  Proving metatheorems with Twelf,
  Representing the judgements of the natural numbers,
  https://twelf.org/wiki/proving-metatheorems-representing-the-judgements-of-the-natural-numbers/,
  Дата обращения: 30.09.24
\end{thebibliography}

\end{document}
